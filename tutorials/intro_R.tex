 \documentclass{beamer}
\usepackage[latin1]{inputenc}

\usepackage{amscd}
\usetheme{Warsaw}

\title{An introduction to R}
\author{Xiang Ao}\institute{HBS}

\usepackage{/usr/share/R/share/texmf/Sweave}
\begin{document}

\begin{frame}
\titlepage
\end{frame}

\begin{frame}
\frametitle{What is R}
\begin{itemize}
\item R is an Open Source system originally written by Ross Ihaka and
Robert Gentleman at the University of Auckland in about 1994.
\item R is based on S language; S is a language for ``programming with data".
John Chambers of Bell Labs has been its main developer for more than
two decades. \item r-project.org
\end{itemize}
\end{frame}

\begin{frame}
\frametitle{Why R}
\begin{itemize}
\item Statisticians use R.  Some econometricians are starting to use
it.
\item It's a flexible statistical programming environment.
\item It has lots of user-written packages, functions which are relatively easy to modify.
\item It's free.
\end{itemize}
\end{frame}

\begin{frame}
\frametitle{Reading in data}
\begin{itemize}
\item Statisticians use R.
\item It's a flexible statistical programming environment.
\item It has lots of user-written packages, functions which are relatively easy to modify.
\item It's free.
\end{itemize}
\end{frame}


\begin{frame}
\frametitle{Workspace}
\begin{itemize}
\item ls() lists all objects currently in your workspace
\item rm(object) removes object from your workspace
\item save(object, file=path/file) saves an object to a file
\item load(path/file) loads an object from a file
\item save.image() saves your workspace to a file called .RData in your working directory. Happens also if you type q("yes").
\item getwd() shows the path of your current working directory
\item setwd(path) allows you to a new path for your current working directory
\end{itemize}

\end{frame}


\end{document}
